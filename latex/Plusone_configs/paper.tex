% Template for PLoS
% Version 3.6 Aug 2022
%
% % % % % % % % % % % % % % % % % % % % % %
%
% -- IMPORTANT NOTE
%
% This template contains comments intended 
% to minimize problems and delays during our production 
% process. Please follow the template instructions
% whenever possible.
%
% % % % % % % % % % % % % % % % % % % % % % % 
%
% Once your paper is accepted for publication, 
% PLEASE REMOVE ALL TRACKED CHANGES in this file 
% and leave only the final text of your manuscript. 
% PLOS recommends the use of latexdiff to track changes during review, as this will help to maintain a clean tex file.
% Visit https://www.ctan.org/pkg/latexdiff?lang=en for info or contact us at latex@plos.org.
%
%
% There are no restrictions on package use within the LaTeX files except that no packages listed in the template may be deleted.
%
% Please do not include colors or graphics in the text.
%
% The manuscript LaTeX source should be contained within a single file (do not use \input, \externaldocument, or similar commands).
%
% % % % % % % % % % % % % % % % % % % % % % %
%
% -- FIGURES AND TABLES
%
% Please include tables/figure captions directly after the paragraph where they are first cited in the text.
%
% DO NOT INCLUDE GRAPHICS IN YOUR MANUSCRIPT
% - Figures should be uploaded separately from your manuscript file. 
% - Figures generated using LaTeX should be extracted and removed from the PDF before submission. 
% - Figures containing multiple panels/subfigures must be combined into one image file before submission.
% For figure citations, please use "Fig" instead of "Figure".
% See http://journals.plos.org/plosone/s/figures for PLOS figure guidelines.
%
% Tables should be cell-based and may not contain:
% - spacing/line breaks within cells to alter layout or alignment
% - do not nest tabular environments (no tabular environments within tabular environments)
% - no graphics or colored text (cell background color/shading OK)
% See http://journals.plos.org/plosone/s/tables for table guidelines.
%
% For tables that exceed the width of the text column, use the adjustwidth environment as illustrated in the example table in text below.
%
% % % % % % % % % % % % % % % % % % % % % % % %
%
% -- EQUATIONS, MATH SYMBOLS, SUBSCRIPTS, AND SUPERSCRIPTS
%
% IMPORTANT
% Below are a few tips to help format your equations and other special characters according to our specifications. For more tips to help reduce the possibility of formatting errors during conversion, please see our LaTeX guidelines at http://journals.plos.org/plosone/s/latex
%
% For inline equations, please be sure to include all portions of an equation in the math environment.  For example, x$^2$ is incorrect; this should be formatted as $x^2$ (or $\mathrm{x}^2$ if the romanized font is desired).
%
% Do not include text that is not math in the math environment. For example, CO2 should be written as CO\textsubscript{2} instead of CO$_2$.
%
% Please add line breaks to long display equations when possible in order to fit size of the column. 
%
% For inline equations, please do not include punctuation (commas, etc) within the math environment unless this is part of the equation.
%
% When adding superscript or subscripts outside of brackets/braces, please group using {}.  For example, change "[U(D,E,\gamma)]^2" to "{[U(D,E,\gamma)]}^2". 
%
% Do not use \cal for caligraphic font.  Instead, use \mathcal{}
%
% % % % % % % % % % % % % % % % % % % % % % % % 
%
% Please contact latex@plos.org with any questions.
%
% % % % % % % % % % % % % % % % % % % % % % % %

\documentclass[10pt,letterpaper]{article}
\usepackage[top=0.85in,left=2.75in,footskip=0.75in]{geometry}

% amsmath and amssymb packages, useful for mathematical formulas and symbols
\usepackage{amsmath,amssymb}
\usepackage{algorithm}
\usepackage{algorithmic}
\usepackage{booktabs}
\usepackage{rotating}
% Use adjustwidth environment to exceed column width (see example table in text)
\usepackage{changepage}

% textcomp package and marvosym package for additional characters
\usepackage{textcomp,marvosym}

% cite package, to clean up citations in the main text. Do not remove.
\usepackage{cite}

% Use nameref to cite supporting information files (see Supporting Information section for more info)
\usepackage{nameref,hyperref}

% line numbers
\usepackage[right]{lineno}

% ligatures disabled
\usepackage[nopatch=eqnum]{microtype}
\DisableLigatures[f]{encoding = *, family = * }

% color can be used to apply background shading to table cells only
\usepackage[table]{xcolor}

% array package and thick rules for tables
\usepackage{array}

% create "+" rule type for thick vertical lines
\newcolumntype{+}{!{\vrule width 2pt}}

% create \thickcline for thick horizontal lines of variable length
\newlength\savedwidth
\newcommand\thickcline[1]{%
  \noalign{\global\savedwidth\arrayrulewidth\global\arrayrulewidth 2pt}%
  \cline{#1}%
  \noalign{\vskip\arrayrulewidth}%
  \noalign{\global\arrayrulewidth\savedwidth}%
}

% \thickhline command for thick horizontal lines that span the table
\newcommand\thickhline{\noalign{\global\savedwidth\arrayrulewidth\global\arrayrulewidth 2pt}%
\hline
\noalign{\global\arrayrulewidth\savedwidth}}


% Remove comment for double spacing
%\usepackage{setspace}
%\doublespacing

% Text layout
\raggedright
\setlength{\parindent}{0.5cm}
\textwidth 5.25in
\textheight 8.75in

% Bold the 'Figure #' in the caption and separate it from the title/caption with a period
% Captions will be left justified
\usepackage[aboveskip=1pt,labelfont=bf,labelsep=period,justification=raggedright,singlelinecheck=off]{caption}
\renewcommand{\figurename}{Fig}

% Use the PLoS provided BiBTeX style
\bibliographystyle{plos2015}

% Remove brackets from numbering in List of References
\makeatletter
\renewcommand{\@biblabel}[1]{\quad#1.}
\makeatother



% Header and Footer with logo
\usepackage{lastpage,fancyhdr,graphicx}
\usepackage{epstopdf}
%\pagestyle{myheadings}
\pagestyle{fancy}
\fancyhf{}
%\setlength{\headheight}{27.023pt}
%\lhead{\includegraphics[width=2.0in]{PLOS-submission.eps}}
\rfoot{\thepage/\pageref{LastPage}}
\renewcommand{\headrulewidth}{0pt}
\renewcommand{\footrule}{\hrule height 2pt \vspace{2mm}}
\fancyheadoffset[L]{2.25in}
\fancyfootoffset[L]{2.25in}
\lfoot{\today}

%% Include all macros below

\newcommand{\lorem}{{\bf LOREM}}
\newcommand{\ipsum}{{\bf IPSUM}}

%% END MACROS SECTION


\begin{document}
\vspace*{0.2in}

% Title must be 250 characters or less.
\begin{flushleft}
{\Large
\textbf\newline{Machine Learning Trading Strategy in VIX Derivatives:\\{\textit{A Walk-Forward Training and Backtesting Study
}}} % Please use "sentence case" for title and headings (capitalize only the first word in a title (or heading), the first word in a subtitle (or subheading), and any proper nouns).
}
\newline
% Insert author names, affiliations and corresponding author email (do not include titles, positions, or degrees).
\\
Name1 Surname\textsuperscript{1,2\Yinyang},
Name2 Surname\textsuperscript{2\Yinyang},
Name3 Surname\textsuperscript{2,3\textcurrency},
Name4 Surname\textsuperscript{2},
Name5 Surname\textsuperscript{2\ddag},
Name6 Surname\textsuperscript{2\ddag},
Name7 Surname\textsuperscript{1,2,3*},
with the Lorem Ipsum Consortium\textsuperscript{\textpilcrow}
\\
\bigskip
\textbf{1} Affiliation Dept/Program/Center, Institution Name, City, State, Country
\\
\textbf{2} Affiliation Dept/Program/Center, Institution Name, City, State, Country
\\
\textbf{3} Affiliation Dept/Program/Center, Institution Name, City, State, Country
\\
\bigskip

% Insert additional author notes using the symbols described below. Insert symbol callouts after author names as necessary.
%
% Remove or comment out the author notes below if they aren't used.
%
% Primary Equal Contribution Note
\Yinyang These authors contributed equally to this work.

% Additional Equal Contribution Note
% Also use this double-dagger symbol for special authorship notes, such as senior authorship.
\ddag These authors also contributed equally to this work.

% Current address notes
\textcurrency Current Address: Dept/Program/Center, Institution Name, City, State, Country % change symbol to "\textcurrency a" if more than one current address note
% \textcurrency b Insert second current address
% \textcurrency c Insert third current address

% Deceased author note
\dag Deceased

% Group/Consortium Author Note
\textpilcrow Membership list can be found in the Acknowledgments section.

% Use the asterisk to denote corresponding authorship and provide email address in note below.
* correspondingauthor@institute.edu

\end{flushleft}
% Please keep the abstract below 300 words
\section*{Abstract}
Lorem ipsum dolor sit amet, consectetur adipiscing elit. Curabitur eget porta erat. Morbi consectetur est vel gravida pretium. Suspendisse ut dui eu ante cursus gravida non sed sem. Nullam sapien tellus, commodo id velit id, eleifend volutpat quam. Phasellus mauris velit, dapibus finibus elementum vel, pulvinar non tellus. Nunc pellentesque pretium diam, quis maximus dolor faucibus id. Nunc convallis sodales ante, ut ullamcorper est egestas vitae. Nam sit amet enim ultrices, ultrices elit pulvinar, volutpat risus.


% Please keep the Author Summary between 150 and 200 words
% Use first person. PLOS ONE authors please skip this step.
% Author Summary not valid for PLOS ONE submissions.

% Use "Eq" instead of "Equation" for equation citations.
\section*{Introduction}
Lorem ipsum dolor sit~\cite{bib1} amet, consectetur adipiscing elit. Curabitur eget porta erat. Morbi consectetur est vel gravida pretium. Suspendisse ut dui eu ante cursus gravida non sed sem. Nullam Eq~(\ref{eq:schemeP}) sapien tellus, commodo id velit id, eleifend volutpat quam. Phasellus mauris velit, dapibus finibus elementum vel, pulvinar non tellus. Nunc pellentesque pretium diam, quis maximus dolor faucibus id.~\cite{bib2} Nunc convallis sodales ante, ut ullamcorper est egestas vitae. Nam sit amet enim ultrices, ultrices elit pulvinar, volutpat risus.

\begin{eqnarray}
\label{eq:XGBoost}
	\mathrm{P_Y} = \underbrace{H(Y_n) - H(Y_n|\mathbf{V}^{Y}_{n})}_{S_Y} + \underbrace{H(Y_n|\mathbf{V}^{Y}_{n})- H(Y_n|\mathbf{V}^{X,Y}_{n})}_{T_{X\rightarrow Y}},
\end{eqnarray}

\section*{Research Methodology}
\subsection*{VIX term structures decompositions}
The VIX is calculated by the Chicago Board Options Exchange (CBOE) and is based on the prices of options on the S\&P 500 Index.
\\Let $t$ denote time and let $VIX_t$ denote the value of VIX on that date. At time $t$,
we can view lots of VIX futures contracts with different expiration dates:
\begin{eqnarray}
    {F^i_t} := \text{VIX futures expiring at time $T_i$ at time t}
\end{eqnarray}
where $i$ denote the index of the VIX futures contracts with a ascending expiration dates $T_1 < T_2 < ... < T_d$.
\\A term-structure of constant-maturity VIX futures(CMFs), consist of 1,2,...,6 months in our case, are constructed as
a linear interpolation of two nearest expiration dates VIX futures:
\begin{eqnarray}
\label{eq:defineV}
    {V^j_t} := \omega^j_tF^{L_j}_t\ + (1 - \omega^j_t)F^{R_j}_t \text{for j in \{1,2,...,6\}}
\end{eqnarray}
where $L_j$ represents the VIX futures of the nearest expiration dates close to $j$-month on the left of time axis,
$R_j$ represents the VIX futures of the nearest expiration dates close to $j$-month on the right of time axis, and
$\omega^j_t = \frac{T_{R_j} - 30j}{T_{R_j} - T_{L_j}}$. Note that $VIX_t$ is like a zero-horizon CMF.
CMFs are commonly used for they do not suffer fluctuations caused by contract expiry.

\subsubsection*{Rolling VIX Futures Derivatives and Strategies}
Through a rolling VIX futures derivatives or strategies can reappearance the daily returns of CMFs.
SPVXSP, SPVIX2ME, SPVIX3ME, SPVIX4ME, SPVXMP, SPVIX6ME represents 1-6 months CMFs, they can be traded or replicated with trades in the rolling VIX futures strategies.
Derivatives like ETFs and ETNs or Strategies maintains the CMF weights of equation \ref{eq:define V} for fixed maturity $j-month$. For each $j$, we denote $R^j$
the daily return of rolling VIX futures strategy, trading cost is not considered:
\begin{eqnarray}
    {R^j_t} = \frac{\omega^j_t{\Delta}F^{L_j}_t\ + (1 - \omega^j_t){\Delta}F^{R_j}_t}{\omega^j_tF^{L_j}_t\ + (1 - \omega^j_t)F^{R_j}_t}
\end{eqnarray}
where${\Delta}F^{L_j}_t = F^{L_j}_{t+1} - F^{L_j}_t$. We denote ${\Delta}t = \frac{1}{252}$, transform equation in terms of the CMFs:
\begin{eqnarray}
\label{eq: $R^j_t$2}
    {R^j_t} = \frac{{\Delta}V^j_t}{V^j_t} - \frac{F^{R_j}_{t+1} - F^{L_j}_{t+1}}{V^j_t(T_{R_j} - T_{L_j}){\Delta}t}{\Delta}t
\end{eqnarray}
Let Roll denote the drift term in equation \ref{eq: $R^j_t$2}, it referred to as the roll yield of rolling VIX futures strategy:
\begin{eqnarray}
    {Roll^j_{t+1}} := - \frac{F^{R_j}_{t+1} - F^{L_j}_{t+1}}{V^j_t(T_{R_j} - T_{L_j}){\Delta}t}
\end{eqnarray}
In a Contango circumstances, $F^{R_j}_{t+1} > F^{L_j}_{t+1}$ means $Roll < 0$. On the opposite, in a Backwardation circumstances $Roll > 0$.
We re-write equation as follows:
\begin{eqnarray}
\label{eq: $R^j_t$3}
    {R^j_t} = \frac{{\Delta}V^j_t}{V^j_t} + Roll^j_{t+1}{\Delta}t
\end{eqnarray}
From equation \ref{eq: $R^j_t$3} we see $R^j_t$ consist of two parts, the change in $V^j_t$ and roll yield. As history data shown, the most likely VIX futures
curves are contango, so the value of the rolling VIX futures strategies decay.
\\xxx shows that volatility is a mean-reversion process, in a ideal world, suppose $V^j_t$ and CMFs are both stable for a long time like a month, then
the return of $V^j_t$ should be $V^j_t - V^{j-1}_t$, so we denote $\mu$ as :
\begin{eqnarray}
\label{eq: mu}
    {\mu^j_t} = V^j_t - V^{j-1}_t
\end{eqnarray}
$\mu^j_t$ contains instantaneous market views of return about $V^j_t$.
ynorm



% For figure citations, please use "Fig" instead of "Figure".
Nulla mi mi, Fig~\ref{fig1} venenatis sed ipsum varius, volutpat euismod diam. Proin rutrum vel massa non gravida. Quisque tempor sem et dignissim rutrum. Lorem ipsum dolor sit amet, consectetur adipiscing elit. Morbi at justo vitae nulla elementum commodo eu id massa. In vitae diam ac augue semper tincidunt eu ut eros. Fusce fringilla erat porttitor lectus cursus, \nameref{S1_Video} vel sagittis arcu lobortis. Aliquam in enim semper, aliquam massa id, cursus neque. Praesent faucibus semper libero.

% Place figure captions after the first paragraph in which they are cited.
\begin{figure}[!h]
\caption{{\bf Bold the figure title.}
Figure caption text here, please use this space for the figure panel descriptions instead of using subfigure commands. A: Lorem ipsum dolor sit amet. B: Consectetur adipiscing elit.}
\label{fig1}
\end{figure}

\subsection*{machine learning models}

\subsubsection*{2.1 XGBoost}
 The concept of boosting, proposed by Freund and Schapire (1999), revolves around an ensemble model strategy that combines multiple weak learners to create a more robust learner. Friedman (1999) introduced Gradient Tree Boosting, an improvement in boosting that incorporates the gradient descent algorithm into the loss function to minimize errors. Expanding on gradient boosting, Chen and Guestrin (2016) introduce XGBoost, a scalable end-to-end tree boosting system based on additive training strategy that augments gradient tree boosting with regularization, efficient handling of sparse data, and enhanced computational efficiency, resulting in superior speed and accuracy. Eq.1 denotes the objective function of XGBoost where \(\mathbf{\sum_{i=1}^{n}l(y_i,\hat{y}_i)}\) represents the training loss function and \(\mathbf{{\sum_{k=1}^{K}}\Omega(f_k)}\) denotes the complexity of trees.
\begin{eqnarray}
\label{eq:schemeP}
\mathrm{Obj}=\underbrace{\mathbf{\sum_{i=1}^{n}l(y_i,\hat{y}_i)}}_{Training \: Loss} +\underbrace{\mathbf{{\sum_{k=1}^{K}}\Omega(f_k)}}_{Complexity \: of \: trees}
\end{eqnarray}


\subsubsection*{2.2 Multilayer Perceptron (MLP)}

(introduction of MLP).
In our experiment, we use leaky ReLU introduced by Mass (2013) as the activation function \(\phi\) for MLP, the mathematical expression of leaky ReLU  is shown in Eq.2. Given activation function \(\phi\), features \(X \in R^ {n\times d}\) where \(x\) presents the batch size and \(d\) presents the dimensionalities
\begin{eqnarray}
\label{eq:schemeP}
\mathrm{\phi^{(i)}} =\mathbf{max({W^{(i)}}^{T}x,0)}=
\begin{cases}
\mathbf{{W^{(i)}}^{T}x},\quad &\mathbf{{W^{(i)}}^{T}x}>0\\0, \quad else
\end{cases}
\end{eqnarray}
where \(W^{(i)}\) is the weight vector of \(i_{th}\) hidden unit while \(x\) is the input features. A naive MLP can be illustrated as figure 2.
\begin{figure}[!h]
\includegraphics[width=0.5\linewidth]{MLP.jpg}
\caption{{\bf Naive MLP:} MLP network with one hidden layer}
\label{fig2}
\end{figure}

\subsubsection*{2.3 Recurrent Neural Network (RNN)}
We then implement multiple machine learning models based on RNN architecture which aims to find patterns in sequences of data. The work of Schmidt (2019) provides a comprehensive overview of the basic architecture and functions of RNNs while \textit{Sherstinsky (2023) work focuses on deriving the classical RNN formulation from differential equations and addresses the challenges in training standard RNNs. The transformation of RNN into the "Vanilla LSTM" network is explained through logical arguments, with a detailed description of the LSTM system's equations and entities. }
RNN train a function $f$:
\begin{eqnarray}
\label{eq: RNN}
    {h_{t}, y_t} = f(h_{t - 1}, x_t)
\end{eqnarray}
where $h$ represents hidden layer, it carry sequence information by $t - 1$, $x_t$ is the input at $t$, $f$ outputs result $y_t$ at $t$ and sequence information $h_t$.

\subsubsection*{2.3.1 Long Short-term Memory (LSTM)}
The Long Short-Term Memory (LSTM) model, originally proposed by Hochreiter and Schmidhuber (1997), represents a significant milestone in the field of neural networks, particularly in handling sequential data. LSTMs were designed to overcome the limitations of traditional RNNs, especially issues related to learning long-range dependencies. Traditional RNNs struggled with the vanishing and exploding gradient problems, making it challenging to retain information over long sequences. The unique memory cells and input, forget and output gates mechanism enables LSTM to store important information and forget irrelevant details over long sequences. With the recent innovation of attention mechanism, Wang and Hao (2020) introduced the Attention-based LSTM (ALSTM) model, they integrates a multi-head dot product attention within the LSTM architecture and significantly enhances model ability for complex reasoning over sequences.



Eq \ref {eq: LSTM} and Eq \ref {eq: ALSTM} illustrates the mathematical details for the implementation of LSTM and ALSTM in our study.
LSTM framework denote $c$ as the long-term sequence information distinguish from $h$, and design gates to filter information. A gate cell can be represent as:
\begin{eqnarray}
\label{eq: Gate}
    g = \sigma(W\bullet[h_{t - 1}, x_t] + b)
\end{eqnarray}
where $W, b$ are trainable parameters, $\sigma$ represent activation function, so $g$ can be use as a gate. LSTM framework design three gate on RNN as follow:
\begin{eqnarray}
\label{eq: LSTM}
    \begin {cases} c_t = g^f{\odot}c_{t - 1} - g^i{\odot}tanh(W\bullet[h_{t - 1}, x_t] + b)
    \\h^t = g^o{\odot}tanh(c_t)
    \\y^t = f(h_t)
    \end {cases}
\end{eqnarray}
$g^f, g^i, g^o$ represent three gates, and $f$ is some function link LSTM outputs to downstream task.
\\ALSTM add a encoder and a decoder attention-based layers in LSTM, Given $n$ series, stage one encoder layer can learn attention at a same time and transform to new input.
\begin{eqnarray}
\label{eq: ALSTM1}
    \begin {cases} e^k_t = tanh(W\bullet[h_{t - 1}, c_{t - 1}] + bx^k)
    \\{\alpha}^k_t = \frac{exp(e^k_t)}{\sum_{i=1}^n exp(e^i_t)}
    \\\widetilde{x_t} = ({\alpha^1_t}x^1_t, {\alpha^2_t}x^2_t,..., {\alpha^n_t}x^n_t)
    \end {cases}
\end{eqnarray}
with $\widetilde{x_t}$ as input, stage one construct with a LSTM framework named $L_1$, stage two decoder layer learn attention with all the hidden state $h_t$ of $L_1$,
the output $\widetilde{h_t}$ transform the label $y$ of another LSTM framework named $L_2$, so two attention layers can learn relations between $n$ series.
Let $d$ denote the hidden state of $L_2$, transform of $y$ can be formulate as:
\begin{eqnarray}
\label{eq: ALSTM2}
    \begin {cases} l_t = tanh(W\bullet[d_{t - 1}, c^2_{t - 1}] + bh^i)
    \\{\beta}_t = \frac{exp(l_t)}{\sum_{i=1}^T exp({\beta}_t)}
    \\c_t = \sum_{i=1}^T {\beta}_ih_i
    \\\widetilde{y_{t - 1}} = W\bullet[y_{t - 1}; c_{t - 1}] + b
    \end {cases}
\end{eqnarray}
$\tilde{y}$ is the input of $L_2$.

\subsubsection*{2.3.2 Gated Recurrent Unit}
 The Gated Recurrent Unit (GRU), first introduced by Cho et al. (2014), is designed to capture temporal dependencies in sequential data efficiently. This efficiency in GRU is achieved through its unique architecture, consisting of two gates: the update gate and the reset gate. The update gate plays a crucial role in determining how much of its previous state the GRU retains, enabling the effective capture of long-term dependencies. Conversely, the reset gate influences the amount of past information to be forgotten, assisting the model in concentrating on the most pertinent information. These gates effectively control the flow of information within the unit, allowing the network to retain important information from past data inputs while discarding irrelevant data.This selective memory mechanism empowers the GRU to effectively tackle the vanishing gradient problem, a common challenge in traditional RNNs.The implementation details can be found in eq.4.

\begin{eqnarray}
\label{eq: GRU}
    \begin {cases} h^{'}_{t - 1} = h_{t - 1}{\bullet}g^r
    \\h^{'} = tanh(W\bullet[h^{'}_{t - 1}, x_t] + b)
    \\h^t = (1 - g^z){\odot}tanh(c_t) + g^z{\odot}h^{'}
    \end {cases}
\end{eqnarray}
Follow notation in LSTM,  $g$ represents gate \ref{eq: Gate}.



\subsubsection*{2.4 Temporal Covolutional Networks}
Temporal covolutional networks (TCN), originally introduced by Lea et al. (2016), is a type of neural network that employs convolutional layers to capture dependencies in sequential data, . Unlike traditional RNN and LSTMs, TCNs use 1D convolutions to process input sequences in parallel, making them highly parallelizable and capable of capturing long-range dependencies. Implementation details can be found in eq.5.

\begin{eqnarray}
\label{eq: TCN}
    TCN = FCN + \text{causal convolutions}
\end{eqnarray}
where FCN is fully-convolutional network with one dimension.


\subsubsection* {2.5 Transformer}
We also implement attention-based network architecture called transformer proposed by Vaswani et al. (2017) for prediction. Transformer departs from the traditional recurrent or convolutional layers, relying instead on the attention mechanism, This enables the model to weigh the significance of different parts of the input data differently, allowing it to capture complex dependencies and relationships within the data. Eq.6 illustrates the implementation details.

\begin{eqnarray}
\label{eq: Transformer}
    y_i = F(encoder(X), decoder(y_1,...,y_{i - 1}))
\end{eqnarray}
In our case, we only use encoder to produce output, for we don't need to produce output one by one. Encoder's framework follows by:
\begin{eqnarray}
\label{eq: TransformerDecoder}
    \begin {cases} \tilde{X} = \text{Positional Encoding} + X
    \\Y = FFN(\text{self-attention}(X))
    \end {cases}
\end{eqnarray}
where FFN is a feedforward network with residual added and layer norm, self-attention's framework follows by:
\begin{eqnarray}
\label{eq: self-attention}
    \begin {cases} Q = W^QX
    \\K = W^KX
    \\V = W^VX
    \\Z = \test{softmax}(QK^T)V
    \end {cases}
\end{eqnarray}
where $Z$ is the output of self-attention layer.

\section*{data and experiment design}
\subsubsection*{Data Preparations}
In line with our primary goal of predicting next-day arithmetical return of \(CMFs\) as denoted in eq.X, it is rational to initially incorporate CMFs-related factors \(V^j_t\) and \(Roll^j_{t+1}\) as detailed in Eq.2 and Eq.5. Additionally, to thoroughly examine the informative power of term structures, we introduce \(\mu^j_t\) denoted in Eq.7 and \(\triangle Roll_{t}^j\)to provides a comprehensive description of the CMFs term structure. We also integrate \(TLT\) and \(SPY\) into our analysis to partially represents the macro-economics conditions. Furthermore, various time-series statistical transformations are employed to explore possible extra information with details provided in table 3. The time range for \(VIX\)  futures and \(CMFs\)  data covers the period from January 3, 2005, to March 7, 2023 while \(TLT\)  and \(SPY\)  data are selected within the time frame of January 12, 1995, to April 12, 2023. Three distinct data sets are utilised to incrementally examine the informative power of VIX term structures and additional information offered by time-series statistical transformations.
Detailed descriptions of all three datasets are provided in tables 1, 2 and 3.
\begin{table}[!ht]
%\begin{adjustwidth}{-2.25in}{0in} % Comment out/remove adjustwidth environment if table fits in text column.
\centering
\caption{
{\bf Simple Data sets}}
\begin{tabular}{|c|c|c|} % Use c for centered columns; you can also use other alignment options like l (left) or r (right)
    \hline % Horizontal line at the top
    \(VIX_{CMFs}\)  & CMFs-related factors & Macro-factors \\
    \hline % Horizontal line under column headers
    \(CMF_{1m}\)  & \({[V_{t}^1},{Roll_{t}^1]}\) & \([log_{SPY_t},log_{TLT_t},VIX_t]\) \\
    \hline
    \(CMF_{2m}\)  & \({[V_{t}^2},{Roll_{t}^2]}\)  & \([log_{SPY_t},log_{TLT_t},VIX_t]\) \\
    \hline
    \(CMF_{3m}\)  & \({[V_{t}^3},{Roll_{t}^3]}\) & \([log_{SPY_t},log_{TLT_t},VIX_t]\) \\
    \hline
    \(CMF_{4m}\)& \({[V_{t}^4},{Roll_{t}^4]}\) & \([log_{SPY_t},log_{TLT_t},VIX_t]\)\\
    \hline
    \(CMF_{5m}\) &  \({[V_{t}^5},{Roll_{t}^5]}\) & \([log_{SPY_t},log_{TLT_t},VIX_t]\) \\
    \hline
    \(CMF_{6m}\) & \({[V_{t}^6},{Roll_{t}^6]}\) & \([log_{SPY_t},log_{TLT_t},VIX_t]\) \\
    \hline % Horizontal line at the bottom
  \end{tabular}
\begin{flushleft} Table notes: The initial data sets comprise solely of CMFs-related factors and macroeconomic variables.
\end{flushleft}
\label{table1}
%\end{adjustwidth}
\end{table}
%%%%%%%%%%%%%%%%%%%%%%%%%%%%%%%%%%%%%%%%%%%%%%%%%
\begin{table}[!ht]
%\begin{adjustwidth}{-2.25in}{0in} % Comment out/remove adjustwidth environment if table fits in text column.
\centering
\caption{
{\bf Term Structure Data sets}}
\begin{tabular}{|c|c|c|c|c|c|} % Use c for centered columns; you can also use other alignment options like l (left) or r (right)
    \hline % Horizontal line at the top
    \(VIX_{CMFs}\)  & CMFs-related factors & Macro-factors & Term Structures factors \\
    \hline % Horizontal line under column headers
    \(CMF_{1m}\)  & \({[V_{t}^1},{Roll_{t}^1]}\) & \([log_{SPY_t},log_{TLT_t},VIX_t]\) & \({[\mu_{t}^1, \triangle Roll_{t}^1]}\)  \\
    \hline
    \(CMF_{2m}\)  & \({[V_{t}^2},{Roll_{t}^2]}\)  & \([log_{SPY_t},log_{TLT_t},VIX_t]\) &  \({[\mu_{t}^2, \triangle Roll_{t}^2]}\) \\
    \hline
    \(CMF_{3m}\)  & \({[V_{t}^3},{Roll_{t}^3]}\) & \([log_{SPY_t},log_{TLT_t},VIX_t]\) & \({[\mu_{t}^3, \triangle Roll_{t}^3]}\) \\
    \hline
    \(CMF_{4m}\)& \({[V_{t}^4},{Roll_{t}^4]}\) & \([log_{SPY_t},log_{TLT_t},VIX_t]\) & \({[\mu_{t}^4, \triangle Roll_{t}^4]}\)\\
    \hline
    \(CMF_{5m}\) &  \({[V_{t}^5},{Roll_{t}^5]}\) & \([log_{SPY_t},log_{TLT_t},VIX_t]\) & \({[\mu_{t}^5, \triangle Roll_{t}^5]}\) \\
    \hline
    \(CMF_{6m}\) & \({[V_{t}^6},{Roll_{t}^6]}\) & \([log_{SPY_t},log_{TLT_t},VIX_t]\) & \({[\mu_{t}^6, \triangle Roll_{t}^6]}\)  \\
    \hline % Horizontal line at the bottom
  \end{tabular}
\begin{flushleft} Table notes: Term structures have been added to the data sets based on the simple datasets.
\end{flushleft}
\label{table1}
%\end{adjustwidth}
\end{table}
%%%%%%%%%%%%%%%%%%%%%%%%%%%%%%%%%%%%%%%
\begin{sidewaystable}
\caption{{\bf Statistical Derivations Data Sets}}
%\begin{adjustwidth}{2.25in}{0in} % Comment out/remove adjustwidth environment if table fits in text column.
\centering
\begin{tabular}{|c|c|c|c|c|c|}% Use c for centered columns; you can also use other alignment options like l (left) or r (right)
    \hline % Horizontal line at the top
    \(VIX_{CMFs}\)  & CMFs-related Factors & Macro-factors & Term Structure Factors & Time-series Statistical Derivations\\
    \hline % Horizontal line under column headers
    \(CMF_{1}\)  & \({[V_{t}^1},{Roll_{t}^1]}\) & \([log_{SPY_t},log_{TLT_t},VIX_t]\) & \({[\mu_{t}^1, \triangle Roll_{t}^1]}\) &
    \makecell[c]{\([Std_{T}(All_{t}^{1}), Skew_{T}^{1}(All_{t}^{1}),Kurt_T(All_{t}^{1}),Mean_{T}(All_{t}^{1})\),\\ \(z\ score_{T}(term_{t}^{1})\)\\ , \(where \: T \in \{5,20, 60\}\)]}   \\
    \hline
    \(CMF_{2}\)  & \({[V_{t}^2},{Roll_{t}^2]}\)  & \([log_{SPY_t},log_{TLT_t},VIX_t]\) &  \({[\mu_{t}^2, \triangle Roll_{t}^2]}\) &   \makecell[c]{\([Std_{T}(All_{t}^{2}), Skew_{T}^{2}(All_{t}^{2}),Kurt_T(All_{t}^{2}),Mean_{T}(All_{t}^{2})\),\\ \(z\ score_{T}(term_{t}^{2})\)\\ , \(where \: T \in \{5,20, 60\}\)]} \\
    \hline
    \(CMF_{3}\)  & \({[V_{t}^3},{Roll_{t}^3]}\) & \([log_{SPY_t},log_{TLT_t},VIX_t]\) & \({[\mu_{t}^3, \triangle Roll_{t}^3]}\) &  \makecell[c]{\([Std_{T}(All_{t}^{3}), Skew_{T}^{3}(All_{t}^{3}),Kurt_T(All_{t}^{3}),Mean_{T}(All_{t}^{3})\),\\ \(z\ score_{T}(term_{t}^{3})\)\\ , \(where \: T \in \{5,20, 60\}\)]}  \\
    \hline
    \(CMF_{4}\)& \({[V_{t}^4},{Roll_{t}^4]}\) & \([log_{SPY_t},log_{TLT_t},VIX_t]\) & \({[\mu_{t}^4, \triangle Roll_{t}^4]}\) &   \makecell[c]{\([Std_{T}(All_{t}^{4}), Skew_{T}^{4}(All_{t}^{4}),Kurt_T(All_{t}^{4}),Mean_{T}(All_{t}^{4})\),\\ \(z\ score_{T}(term_{t}^{4})\)\\ , \(where \: T \in \{5,20, 60\}\)]} \\
    \hline
    \(CMF_{5}\) &  \({[V_{t}^5},{Roll_{t}^5]}\) & \([log_{SPY_t},log_{TLT_t},VIX_t]\) & \({[\mu_{t}^5, \triangle Roll_{t}^5]}\) &  \makecell[c]{\([Std_{T}(All_{t}^{5}), Skew_{T}^{5}(All_{t}^{5}),Kurt_T(All_{t}^{5}),Mean_{T}(All_{t}^{5})\),\\ \(z\ score_{T}(term_{t}^{5})\)\\ , \(where \: T \in \{5,20, 60\}\)]} \\
    \hline
    \(CMF_{6}\) & \({[V_{t}^6},{Roll_{t}^6]}\) & \([log_{SPY_t},log_{TLT_t},VIX_t]\) & \({[\mu_{t}^6, \triangle Roll_{t}^6]}\) & \makecell[c]{\([Std_{T}(All_{t}^{6}), Skew_{T}^{6}(All_{t}^{6}),Kurt_T(All_{t}^{6}),Mean_{T}(All_{t}^{6})\),\\ \(z\ score_{T}(term_{t}^{6})\)\\ , \(where \: T \in \{5,20, 60\}\)]}  \\
    \hline % Horizontal line at the bottom
\end{tabular}
\begin{flushleft} Table notes: \( All_{t}^{j} \in \{CMFs\ related \ Factors, Macro \ factors, Term\ Structure\ Factors\}\) and \(term_{t}^{j} \in \{Term\ Structure\ Factors \}\)

\end{flushleft}
\label{table1}
%\end{adjustwidth}
\end{sidewaystable}

 \subsubsection*{Empirical Experiment Designs}
 The main goal of our empirical experiment aims to evaluate the statistical and economic performances among different models and different datasets.
Before training we are dealing with tabular datasets with multi-index datetime and instruments, columns represents features. we form dataset {(Xi, yi)}i=1,2,..,I into a single combined dataset(X, y) to train model instead of train I models every task named "features stacking". With this dataset organization, conclusions are more general and less biased by outliers or over fitting and model can be trained with more data, correspondingly correlation of instruments are not considered in the model.



\subsubsection*{Trading Strategy: A mean-variance approach}
To ascertain optimal portfolio weights grounded in forecasted data, we employed mean-variance optimization incorporating additional constraints. The covariance structure was derived through the estimation of a 60-day historical covariance matrix. The risk aversion parameter, denoted as gamma, was fixed at a value of 0.2.
To ensure prudent risk management, we imposed constraints on portfolio characteristics. Specifically, we restricted gross leverage to a maximum of 3, and imposed absolute limits on the individual weights of VIX futures indices, ensuring they did not exceed an absolute value of 1. Furthermore, a critical risk threshold was established, with the annualized risk of the portfolio constrained to not exceed 30%.
This methodology, integrating forecast-driven optimization within a risk-constrained mean-variance framework, provides a rigorous approach to portfolio construction, striking a balance between expected returns and risk mitigation. The application of historical covariance data and the delineation of precise risk parameters contribute to the robustness and reliability of the derived portfolio weights, thereby enhancing the integrity of the investment strategy.













Nulla mi mi, venenatis sed ipsum varius, Table~\ref{table1} volutpat euismod diam. Proin rutrum vel massa non gravida. Quisque tempor sem et dignissim rutrum. Lorem ipsum dolor sit amet, consectetur adipiscing elit. Morbi at justo vitae nulla elementum commodo eu id massa. In vitae diam ac augue semper tincidunt eu ut eros. Fusce fringilla erat porttitor lectus cursus, vel sagittis arcu lobortis. Aliquam in enim semper, aliquam massa id, cursus neque. Praesent faucibus semper libero.

% Place tables after the first paragraph in which they are cited.
\begin{table}[!ht]
\begin{adjustwidth}{-2.25in}{0in} % Comment out/remove adjustwidth environment if table fits in text column.
\centering
\caption{all result}
% \begin{sidewaystable}
\begin{tabular}{llrrrrp{1.5cm}p{1.5cm}p{1.5cm}p{1.5cm}}
\toprule
    &               &     IC &   ICIR &  Rank IC &  Rank ICIR &  C-MVO Annualized Return &  C-MVO Information Ratio &  Long Short Annualized Return &  Long Short Information Ratio \\
model & dataset &        &        &          &            &                          &                          &                               &                               \\
\midrule
ALSTM & Derivations &  0.040 &  0.066 &    0.027 &      0.048 &                   -0.002 &                   -0.016 &                         0.033 &                         0.174 \\
    & Simple &  0.005 &  0.007 &   -0.005 &     -0.008 &                    0.057 &                    0.803 &                         0.051 &                         0.258 \\
    & TermStructure &  0.029 &  0.042 &    0.026 &      0.041 &                    0.071 &                    0.744 &                         0.112 &                         0.579 \\
\midrule
GRU & Derivations &  0.021 &  0.035 &    0.017 &      0.031 &                    0.049 &                    0.310 &                         0.096 &                         0.641 \\
    & Simple & -0.023 & -0.033 &   -0.026 &     -0.040 &                    0.045 &                    0.517 &                         0.059 &                         0.310 \\
    & TermStructure &  0.004 &  0.006 &    0.002 &      0.004 &                    0.028 &                    0.241 &                         0.080 &                         0.409 \\
\midrule
LGB & Derivations &  0.019 &  0.059 &    0.085 &      0.127 &                    0.020 &                    0.264 &                        -0.085 &                        -0.355 \\
    & Simple & -0.024 & -0.050 &   -0.044 &     -0.069 &                   -0.004 &                   -0.044 &                        -0.121 &                        -0.511 \\
    & TermStructure &  0.035 &  0.066 &    0.054 &      0.079 &                    0.023 &                    0.327 &                        -0.085 &                        -0.367 \\
\midrule
LSTM & Derivations &  0.040 &  0.067 &    0.036 &      0.064 &                    0.047 &                    0.416 &                         0.020 &                         0.123 \\
    & Simple & -0.015 & -0.022 &   -0.026 &     -0.042 &                    0.030 &                    0.388 &                         0.044 &                         0.262 \\
    & TermStructure &  0.021 &  0.031 &    0.016 &      0.026 &                    0.059 &                    0.671 &                         0.034 &                         0.168 \\
\midrule
Linear & Derivations &  0.095 &  0.147 &    0.079 &      0.140 &                    0.165 &                    1.049 &                         0.117 &                         0.646 \\
    & Simple &  0.090 &  0.147 &    0.070 &      0.131 &                    0.068 &                    1.325 &                         0.186 &                         0.992 \\
    & TermStructure &  0.114 &  0.162 &    0.091 &      0.152 &                    0.150 &                    2.291 &                         0.214 &                         1.105 \\
\midrule
MLP & Derivations & -0.011 & -0.018 &   -0.013 &     -0.024 &                   -0.023 &                   -0.074 &                         0.056 &                         0.325 \\
    & Simple & -0.002 & -0.004 &   -0.001 &     -0.001 &                    0.195 &                    0.630 &                        -0.009 &                        -0.046 \\
    & TermStructure &  0.011 &  0.017 &    0.012 &      0.020 &                    0.013 &                    0.043 &                         0.050 &                         0.302 \\
\midrule
XGB & Derivations &  0.008 &  0.012 &    0.011 &      0.019 &                    0.062 &                    0.250 &                         0.024 &                         0.141 \\
    & Simple &  0.015 &  0.023 &    0.003 &      0.005 &                   -0.019 &                   -0.073 &                         0.042 &                         0.260 \\
    & TermStructure &  0.047 &  0.071 &    0.043 &      0.069 &                    0.011 &                    0.044 &                         0.116 &                         0.632 \\
\bottomrule
\end{tabular}



\begin{flushleft} Table notes Phasellus venenatis, tortor nec vestibulum mattis, massa tortor interdum felis, nec pellentesque metus tortor nec nisl. Ut ornare mauris tellus, vel dapibus arcu suscipit sed.
\end{flushleft}
\label{table1}
% \end{sidewaystable}
\end{adjustwidth}

\end{table}


%PLOS does not support heading levels beyond the 3rd (no 4th level headings).
\subsection*{data preparation}

\subsection*{experiment design}

\subsubsection*{trading strategies construction (PO)}
Maecenas convallis mauris sit amet sem ultrices gravida. Etiam eget sapien nibh. Sed ac ipsum eget enim egestas ullamcorper nec euismod ligula. Curabitur fringilla pulvinar lectus consectetur pellentesque. Quisque augue sem, tincidunt sit amet feugiat eget, ullamcorper sed velit. Sed non aliquet felis. Lorem ipsum dolor sit amet, consectetur adipiscing elit. Mauris commodo justo ac dui pretium imperdiet. Sed suscipit iaculis mi at feugiat.

\begin{enumerate}
	\item{react}
	\item{diffuse free particles}
	\item{increment time by dt and go to 1}
\end{enumerate}

\subsubsection*{walking forward procedure}
\begin{algorithm}
        \caption{walking forward procedure}
        \begin{algorithmic}
            \STATE $rollingtasks \gets Algorithm TaskRollGen\ref{Task_Roll_Gen}$
            \FOR{$task \gets rollingtasks$}
            \STATE $model \xleftarrow{fit} task[train]$
            \STATE $pred \xleftarrow{predict} task[test]$
            \STATE $preds \xleftarrow{append} pred$
            \ENDFOR
            \STATE $PRED \xleftarrow{concat} preds \text{\quad \# PRED is a time series cover Task[test]'s horizon}$
            \STATE $metrics \xleftarrow{calculate} PRED$
            \STATE $backtestingresult \gets PRED, strategy \text{\quad \# bar by bar}$
        \end{algorithmic}
\end{algorithm}
\begin{algorithm}
	\renewcommand{\algorithmicrequire}{\textbf{Input:}}
	\renewcommand{\algorithmicensure}{\textbf{Output:}}
	\caption{Task Roll Gen}
	\label{Task_Roll_Gen}
	\begin{algorithmic}
    	\REQUIRE TASK contains \{TRAIN, VALID, TEST, roll\_lenth\}
            \ENSURE rollingtasks with their own \{train, valid, test\}
            \STATE $fragments \gets$ Test divided by roll\_lenth
            \FOR{$m \gets fragments$}
            \STATE $task \gets Task$
            \STATE $task[test] \gets m$
            \STATE $task[valid] \gets [start\_date_{test} - 1 - lenth_{Valid}, start\_date_{test} - 1]$
            \STATE $task[train] \gets [start\_date_{TRAIN}, strat\_date_{valid} - 1]$ \text{\quad \# expanding}
            \STATE $rollingtasks \xleftarrow{append} task$
            \ENDFOR
	\end{algorithmic}
\end{algorithm}
\subsubsection*{Evaluation metrics : statistical \& economic}

\subsubsection*{features stacking}

\subsection*{Sed ac quam id nisi malesuada congue}

Nulla mi mi, venenatis sed ipsum varius, volutpat euismod diam. Proin rutrum vel massa non gravida. Quisque tempor sem et dignissim rutrum. Lorem ipsum dolor sit amet, consectetur adipiscing elit. Morbi at justo vitae nulla elementum commodo eu id massa. In vitae diam ac augue semper tincidunt eu ut eros. Fusce fringilla erat porttitor lectus cursus, vel sagittis arcu lobortis. Aliquam in enim semper, aliquam massa id, cursus neque. Praesent faucibus semper libero.

\begin{itemize}
	\item First bulleted item.
	\item Second bulleted item.
	\item Third bulleted item.
\end{itemize}


\subsection*{Analysis Results}
\subsubsection*{Dataset Contains Term Structure Infomation Stand Out}
In our experiment, we explore three distinct datasets named Simple, TermStructure, Derivations as before, our results show that TermStructure and Derivations with term structure infomation stand out.
\\Table~\ref{Best Dataset} select the best dataset for all models on our main metrics: IC, rank IC, information ratio of long short strategy, information ratio of constrined mean-variance optimization strategy.

\begin{table}[!ht]
\begin{adjustwidth}{-2.25in}{0in} % Comment out/remove adjustwidth environment if table fits in text column.
\centering
\caption{Best Dataset}
\label{Best Dataset}
% \begin{sidewaystable}
\begin{tabular}{lllll}
\toprule
{} &             IC &        Rank IC & C-MVO Information Ratio & Long Short Information Ratio \\
model  &                &                &                         &                              \\
\midrule
ALSTM  &    Derivations &    Derivations &                  Simple &                TermStructure \\
GRU    &    Derivations &    Derivations &                  Simple &                  Derivations \\
LGB    &  TermStructure &    Derivations &           TermStructure &                  Derivations \\
LSTM   &    Derivations &    Derivations &           TermStructure &                       Simple \\
Linear &  TermStructure &  TermStructure &           TermStructure &                TermStructure \\
MLP    &  TermStructure &  TermStructure &                  Simple &                  Derivations \\
XGB    &  TermStructure &  TermStructure &             Derivations &                TermStructure \\
\bottomrule
\end{tabular}




\begin{flushleft} The best dataset in all models on main metircs: IC, Rank IC, C-MVO Information Ratio, Long Short Information Ratio.
\end{flushleft}
\label{table1}
% \end{sidewaystable}
\end{adjustwidth}

\end{table}
Data shows Simple dataset doesn't prevail in any model on prediction metrics, the best dataset on backtest metric also main distribute on Term Structure and Derivations. TermStructure is the best dataset at 46\% of the time while Derivations prevail at 39\% of the time.
\\As described before, TermStructure mainly extent term structure features like $\mu$ and $\Delta roll$, Derivations extent time series distribution features like rolling-ma and z-score on TermStructure features. Term structure features do improve the result, but time series distribution features don't improve obviously.
\\We will only discuss TermStructure below.

\subsubsection*{}

\begin{table}[!ht]
\begin{adjustwidth}{-2.25in}{0in} % Comment out/remove adjustwidth environment if table fits in text column.
\centering
\caption{Statistic Metric}
% \begin{sidewaystable}
\begin{tabular}{lrrrr}
\toprule
{} &     IC &   ICIR &  Rank IC &  Rank ICIR \\
model  &        &        &          &            \\
\midrule
ALSTM  &  0.029 &  0.042 &    0.026 &      0.041 \\
GRU    &  0.004 &  0.006 &    0.002 &      0.004 \\
LGB    &  0.035 &  0.066 &    0.054 &      0.079 \\
LSTM   &  0.021 &  0.031 &    0.016 &      0.026 \\
Linear &  0.114 &  0.162 &    0.091 &      0.152 \\
MLP    &  0.011 &  0.017 &    0.012 &      0.020 \\
XGB    &  0.047 &  0.071 &    0.043 &      0.069 \\
mean   &  0.037 &  0.056 &    0.035 &      0.056 \\
\bottomrule
\end{tabular}




\begin{flushleft}
Table notes: performance of statistic metric
\end{flushleft}
\label{table1}
% \end{sidewaystable}
\end{adjustwidth}

\end{table}


\begin{table}[!ht]
\begin{adjustwidth}{-2.25in}{0in} % Comment out/remove adjustwidth environment if table fits in text column.
\centering
\caption{Statistic Metric}
% \begin{sidewaystable}
\begin{tabular}{lp{1.5cm}p{1.5cm}p{1.5cm}p{1.5cm}}
\toprule
{} &  C-MVO Annualized Return &  C-MVO Information Ratio &  Long Short Annualized Return &  Long Short Information Ratio \\
model  &                          &                          &                               &                               \\
\midrule
ALSTM  &                    0.071 &                    0.744 &                         0.112 &                         0.579 \\
GRU    &                    0.028 &                    0.241 &                         0.080 &                         0.409 \\
LGB    &                    0.023 &                    0.327 &                        -0.085 &                        -0.367 \\
LSTM   &                    0.059 &                    0.671 &                         0.034 &                         0.168 \\
Linear &                    0.150 &                    2.291 &                         0.214 &                         1.105 \\
MLP    &                    0.013 &                    0.043 &                         0.050 &                         0.302 \\
XGB    &                    0.011 &                    0.044 &                         0.116 &                         0.632 \\
mean   &                    0.051 &                    0.623 &                         0.074 &                         0.404 \\
\bottomrule
\end{tabular}





\begin{flushleft}
Table notes: performance of backtest metric
\end{flushleft}
\label{table1}
% \end{sidewaystable}
\end{adjustwidth}

\end{table}


\subsubsection*{Models Contrast}



\section*{Discussion}
Nulla mi mi, venenatis sed ipsum varius, Table~\ref{table1} volutpat euismod diam. Proin rutrum vel massa non gravida. Quisque tempor sem et dignissim rutrum. Lorem ipsum dolor sit amet, consectetur adipiscing elit. Morbi at justo vitae nulla elementum commodo eu id massa. In vitae diam ac augue semper tincidunt eu ut eros. Fusce fringilla erat porttitor lectus cursus, vel sagittis arcu lobortis. Aliquam in enim semper, aliquam massa id, cursus neque. Praesent faucibus semper libero~\cite{bib3}.



\begin{table}[!ht]
\begin{adjustwidth}{-2.25in}{0in} % Comment out/remove adjustwidth environment if table fits in text column.
\centering
\caption{Backtest Metric}
% \begin{sidewaystable}
\begin{tabular}{llp{2cm}p{2cm}p{2cm}p{2cm}p{2cm}p{2cm}}
\toprule
    &               &  C-MVO Annualized Return &  C-MVO Information Ratio &  C-MVO Max Drawdown &  Long Short Max Drawdown &  Long Short Annualized Return &  Long Short Information Ratio \\
model & dataset &                          &                          &                     &                          &                               &                               \\
\midrule
ALSTM & Simple &                 0.057287 &                 0.802599 &           -0.202557 &                -0.370155 &                      0.051089 &                      0.258037 \\
    & Statistical Derivations &                -0.001859 &                -0.016439 &           -0.381626 &                -0.392643 &                      0.032920 &                      0.174392 \\
    & TermStructure &                 0.070618 &                 0.743925 &           -0.202290 &                -0.452147 &                      0.111682 &                      0.578975 \\
GRU & Simple &                 0.045304 &                 0.517254 &           -0.232781 &                -0.634396 &                      0.058965 &                      0.310354 \\
    & Statistical Derivations &                 0.049236 &                 0.310389 &           -0.632122 &                -0.304792 &                      0.096182 &                      0.640611 \\
    & TermStructure &                 0.027837 &                 0.241176 &           -0.383124 &                -0.446989 &                      0.079937 &                      0.409258 \\
LGB & Simple &                -0.003863 &                -0.043739 &           -0.218599 &                -1.779844 &                     -0.121349 &                     -0.511463 \\
    & Statistical Derivations &                 0.019886 &                 0.263737 &           -0.194941 &                -1.374026 &                     -0.084887 &                     -0.354627 \\
    & TermStructure &                 0.023292 &                 0.327154 &           -0.166017 &                -1.346228 &                     -0.084790 &                     -0.367168 \\
LSTM & Simple &                 0.029657 &                 0.388443 &           -0.238059 &                -0.367246 &                      0.044252 &                      0.262347 \\
    & Statistical Derivations &                 0.047028 &                 0.416228 &           -0.228143 &                -0.535382 &                      0.019771 &                      0.122704 \\
    & TermStructure &                 0.058512 &                 0.671485 &           -0.270701 &                -0.423780 &                      0.033785 &                      0.167775 \\
Linear & Simple &                 0.068256 &                 1.324847 &           -0.107284 &                -0.182701 &                      0.186216 &                      0.992174 \\
    & Statistical Derivations &                 0.164652 &                 1.048583 &           -0.473651 &                -0.375987 &                      0.117332 &                      0.645722 \\
    & TermStructure &                 0.150301 &                 2.291034 &           -0.085517 &                -0.202081 &                      0.214001 &                      1.105489 \\
MLP & Simple &                 0.194595 &                 0.629803 &           -0.661026 &                -0.919978 &                     -0.008624 &                     -0.045972 \\
    & Statistical Derivations &                -0.023281 &                -0.073628 &           -1.500128 &                -0.803446 &                      0.055949 &                      0.324753 \\
    & TermStructure &                 0.012887 &                 0.043224 &           -1.247869 &                -0.404693 &                      0.049920 &                      0.302452 \\
SFM & Simple &                 0.055921 &                 0.265086 &           -0.607000 &                -0.592480 &                      0.043182 &                      0.215211 \\
    & Statistical Derivations &                -0.005174 &                -0.021433 &           -0.652835 &                -0.680252 &                      0.033137 &                      0.179453 \\
    & TermStructure &                -0.024066 &                -0.108211 &           -1.000576 &                -1.069861 &                     -0.017675 &                     -0.090425 \\
XGB & Simple &                -0.019219 &                -0.072545 &           -1.143294 &                -0.731367 &                      0.042195 &                      0.259511 \\
    & Statistical Derivations &                 0.062245 &                 0.249836 &           -0.953685 &                -0.477849 &                      0.023661 &                      0.141244 \\
    & TermStructure &                 0.011343 &                 0.044457 &           -1.282061 &                -0.419102 &                      0.115907 &                      0.631908 \\
\bottomrule
\end{tabular}




\begin{flushleft} Table notes Phasellus venenatis, tortor nec vestibulum mattis, massa tortor interdum felis, nec pellentesque metus tortor nec nisl. Ut ornare mauris tellus, vel dapibus arcu suscipit sed.
\end{flushleft}
\label{table1}
% \end{sidewaystable}
\end{adjustwidth}

\end{table}


\section*{Conclusion}

CO\textsubscript{2} Maecenas convallis mauris sit amet sem ultrices gravida. Etiam eget sapien nibh. Sed ac ipsum eget enim egestas ullamcorper nec euismod ligula. Curabitur fringilla pulvinar lectus consectetur pellentesque. Quisque augue sem, tincidunt sit amet feugiat eget, ullamcorper sed velit. 

Sed non aliquet felis. Lorem ipsum dolor sit amet, consectetur adipiscing elit. Mauris commodo justo ac dui pretium imperdiet. Sed suscipit iaculis mi at feugiat. Ut neque ipsum, luctus id lacus ut, laoreet scelerisque urna. Phasellus venenatis, tortor nec vestibulum mattis, massa tortor interdum felis, nec pellentesque metus tortor nec nisl. Ut ornare mauris tellus, vel dapibus arcu suscipit sed. Nam condimentum sem eget mollis euismod. Nullam dui urna, gravida venenatis dui et, tincidunt sodales ex. Nunc est dui, sodales sed mauris nec, auctor sagittis leo. Aliquam tincidunt, ex in facilisis elementum, libero lectus luctus est, non vulputate nisl augue at dolor. For more information, see \nameref{S1_Appendix}.

\section*{Supporting information}

% Include only the SI item label in the paragraph heading. Use the \nameref{label} command to cite SI items in the text.
\paragraph*{S1 Fig.}
\label{S1_Fig}
{\bf Bold the title sentence.} Add descriptive text after the title of the item (optional).

\paragraph*{S2 Fig.}
\label{S2_Fig}
{\bf Lorem ipsum.} Maecenas convallis mauris sit amet sem ultrices gravida. Etiam eget sapien nibh. Sed ac ipsum eget enim egestas ullamcorper nec euismod ligula. Curabitur fringilla pulvinar lectus consectetur pellentesque.

\paragraph*{S1 File.}
\label{S1_File}
{\bf Lorem ipsum.}  Maecenas convallis mauris sit amet sem ultrices gravida. Etiam eget sapien nibh. Sed ac ipsum eget enim egestas ullamcorper nec euismod ligula. Curabitur fringilla pulvinar lectus consectetur pellentesque.

\paragraph*{S1 Video.}
\label{S1_Video}
{\bf Lorem ipsum.}  Maecenas convallis mauris sit amet sem ultrices gravida. Etiam eget sapien nibh. Sed ac ipsum eget enim egestas ullamcorper nec euismod ligula. Curabitur fringilla pulvinar lectus consectetur pellentesque.

\paragraph*{S1 Appendix.}
\label{S1_Appendix}
{\bf Lorem ipsum.} Maecenas convallis mauris sit amet sem ultrices gravida. Etiam eget sapien nibh. Sed ac ipsum eget enim egestas ullamcorper nec euismod ligula. Curabitur fringilla pulvinar lectus consectetur pellentesque.

\paragraph*{S1 Table.}
\label{S1_Table}
{\bf Lorem ipsum.} Maecenas convallis mauris sit amet sem ultrices gravida. Etiam eget sapien nibh. Sed ac ipsum eget enim egestas ullamcorper nec euismod ligula. Curabitur fringilla pulvinar lectus consectetur pellentesque.

\section*{Acknowledgments}
Cras egestas velit mauris, eu mollis turpis pellentesque sit amet. Interdum et malesuada fames ac ante ipsum primis in faucibus. Nam id pretium nisi. Sed ac quam id nisi malesuada congue. Sed interdum aliquet augue, at pellentesque quam rhoncus vitae.

\nolinenumbers

% Either type in your references using
% \begin{thebibliography}{}
% \bibitem{}
% Text
% \end{thebibliography}
%
% or
%
% Compile your BiBTeX database using our plos2015.bst
% style file and paste the contents of your .bbl file
% here. See http://journals.plos.org/plosone/s/latex for 
% step-by-step instructions.
% 
\begin{thebibliography}{10}

\bibitem{bib1}
Conant GC, Wolfe KH.
\newblock {{T}urning a hobby into a job: how duplicated genes find new
  functions}.
\newblock Nat Rev Genet. 2008 Dec;9(12):938--950.

\bibitem{bib2}
Ohno S.
\newblock Evolution by gene duplication.
\newblock London: George Alien \& Unwin Ltd. Berlin, Heidelberg and New York:
  Springer-Verlag.; 1970.

\bibitem{bib3}
Magwire MM, Bayer F, Webster CL, Cao C, Jiggins FM.
\newblock {{S}uccessive increases in the resistance of {D}rosophila to viral
  infection through a transposon insertion followed by a {D}uplication}.
\newblock PLoS Genet. 2011 Oct;7(10):e1002337.

\end{thebibliography}



\end{document}

