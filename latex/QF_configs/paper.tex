% rQUFguide.tex
% v2.1 released October 2014

\RequirePackage[2018-12-01]{latexrelease}
\documentclass{rQUF2e}

\usepackage{epstopdf}% To incorporate .eps illustrations using PDFLaTeX, etc.
\usepackage{subfigure}% Support for small, `sub' figures and tables

\theoremstyle{plain}
\newtheorem{theorem}{Theorem}[section]
\newtheorem{corollary}[theorem]{Corollary}
\newtheorem{lemma}[theorem]{Lemma}
\newtheorem{proposition}[theorem]{Proposition}

\theoremstyle{definition}
\newtheorem{definition}{Definition}

\theoremstyle{remark}
\newtheorem{remark}{Remark}

\begin{document}

%\jvol{00} \jnum{00} \jyear{2014} \jmonth{October}

\title{Machine Learning Trading Strategy in VIX Derivatives:\\{\textit{A Walk-Forward Training and Backtesting Study
}}}

\author{Sangyuan Wang, Keran Li}

\maketitle

\begin{abstract}
...
\end{abstract}

\begin{keywords}
VIX derivatives; Machine Learning; Trading Strategy 
\end{keywords}

\begin{classcode}Please provide at least one JEL Classification code\end{classcode}


\section{Introduction}
The shape of the VIX futures curve is informative.\\
...\\
(how do we construct our VIX futures curve features)\\
...\\
(someone use a stationary VIX futures curve model for trading, but we prove we can predict VIX ETFs return directly.)\\
...\\

\subsection{Contrubutions}
The key contributions of this study can be summarised as follows, first, based on the VIX futures term structures decomposation approach proposed by Avellaneda(2021), features with term-structure information, personal information, and major market trends information are included in the datasets. Second, various state-of-art machine learning approaches are introduced to predict the VIX rolling-ETFs next-day returns. Third, two trading strategies are used in the backtesting which are, the long-short strategy based on machine learning ranking and a mean-variance adjusted portfolio optimization strategy.
The results reveal that firstly, the VIX futures curve has the ability to predict the next-day return of VIX ETFs. Thirdly, The economic performance of machine learning models relying on the adjusted mean-variance portfolio optimization strategy generally outperforms that of models based on the long-short strategy.

\subsection{Main results}
a, b, c, d\\
...\\
The remainder of the paper is composed as follows: Section 2 provides a review of the literature, Section 3 introduce VIX and VIX derivatives, explains how parameters are estimated, Section 4 describes the dataset of features and its pre-processing, Section 5 describes Methodology for construct a Machine Learning Trading Strategy, Section 6 analysis results, Section 7 concludes and discuss, Appendix: graph and code.


\section{Literature Review}

\subsection{VIX basic introduction}
...

\subsection{VIX futures curve)(term structures)}
...

\subsection{Machine learning}
...
\subsubsection{Machine learning in time series}
...
\subsubsection{Typical machine learning Algorithms}
XGB, GRU, Transformer, TCN, GATs
\subsection{Portfolio Optimization}
...

\section{VIX and VIX derivatives}
...

\section{Datasets and Features}

\subsection{data descriptions}
...
\subsection{features stacking}
...

\section{Methodology for construct a machine learning trading strategy}
...

\subsection{Machine learning models}
...
\subsection{Walk-Forward training and backtesting procedure}
...
\subsection{Trading strategy: a mean-variance approach}
To ascertain optimal portfolio weights grounded in forecasted data, we employed mean-variance optimization incorporating additional constraints. The covariance structure was derived through the estimation of a 60-day historical covariance matrix. The risk aversion parameter, denoted as gamma, was fixed at a value of 0.2.\\
To ensure prudent risk management, we imposed constraints on portfolio characteristics. Specifically, we restricted gross leverage to a maximum of 3, and imposed absolute limits on the individual weights of VIX futures indices, ensuring they did not exceed an absolute value of 1. Furthermore, a critical risk threshold was established, with the annualized risk of the portfolio constrained to not exceed 30%.\\
This methodology, integrating forecast-driven optimization within a risk-constrained mean-variance framework, provides a rigorous approach to portfolio construction, striking a balance between expected returns and risk mitigation. The application of historical covariance data and the delineation of precise risk parameters contribute to the robustness and reliability of the derived portfolio weights, thereby enhancing the integrity of the investment strategy.

\section{Results Metrics}
...

\section{Concludes and Forward}
...

\end{document}
